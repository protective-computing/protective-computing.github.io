% Options for packages loaded elsewhere
\PassOptionsToPackage{unicode}{hyperref}
\PassOptionsToPackage{hyphens}{url}
\documentclass[
]{article}
\usepackage{xcolor}
\usepackage{amsmath,amssymb}
\setcounter{secnumdepth}{-\maxdimen} % remove section numbering
\usepackage{iftex}
\ifPDFTeX
  \usepackage[T1]{fontenc}
  \usepackage[utf8]{inputenc}
  \usepackage{textcomp} % provide euro and other symbols
\else % if luatex or xetex
  \usepackage{unicode-math} % this also loads fontspec
  \defaultfontfeatures{Scale=MatchLowercase}
  \defaultfontfeatures[\rmfamily]{Ligatures=TeX,Scale=1}
\fi
\usepackage{lmodern}
\ifPDFTeX\else
  % xetex/luatex font selection
\fi
% Use upquote if available, for straight quotes in verbatim environments
\IfFileExists{upquote.sty}{\usepackage{upquote}}{}
\IfFileExists{microtype.sty}{% use microtype if available
  \usepackage[]{microtype}
  \UseMicrotypeSet[protrusion]{basicmath} % disable protrusion for tt fonts
}{}
\makeatletter
\@ifundefined{KOMAClassName}{% if non-KOMA class
  \IfFileExists{parskip.sty}{%
    \usepackage{parskip}
  }{% else
    \setlength{\parindent}{0pt}
    \setlength{\parskip}{6pt plus 2pt minus 1pt}}
}{% if KOMA class
  \KOMAoptions{parskip=half}}
\makeatother
\usepackage{longtable,booktabs,array}
\newcounter{none} % for unnumbered tables
\usepackage{calc} % for calculating minipage widths
% Correct order of tables after \paragraph or \subparagraph
\usepackage{etoolbox}
\makeatletter
\patchcmd\longtable{\par}{\if@noskipsec\mbox{}\fi\par}{}{}
\makeatother
% Allow footnotes in longtable head/foot
\IfFileExists{footnotehyper.sty}{\usepackage{footnotehyper}}{\usepackage{footnote}}
\makesavenoteenv{longtable}
\setlength{\emergencystretch}{3em} % prevent overfull lines
\providecommand{\tightlist}{%
  \setlength{\itemsep}{0pt}\setlength{\parskip}{0pt}}
\usepackage{bookmark}
\IfFileExists{xurl.sty}{\usepackage{xurl}}{} % add URL line breaks if available
\urlstyle{same}
\hypersetup{
  hidelinks,
  pdfcreator={LaTeX via pandoc}}

\author{}
\date{}

\begin{document}

\section{Protective Legitimacy Score (PLS) --- Operational Rubric v1.0
(Draft)}\label{protective-legitimacy-score-pls--operational-rubric-v10-draft}

\subsection{1. Purpose}\label{1-purpose}

Convert principle conformance into an auditable, reproducible score for
implementation review under vulnerability conditions.

\subsection{2. Scoring Model}\label{2-scoring-model}

\begin{itemize}
\tightlist
\item
  Score range: \texttt{0–100}
\item
  Principle scores: \texttt{0–4} each
\item
  Weighted composite with hard fail guards
\end{itemize}

\subsubsection{Principle Weights}\label{principle-weights}

\begin{itemize}
\tightlist
\item
  Reversibility: 16
\item
  Exposure Minimization: 18
\item
  Local Authority: 18
\item
  Coercion Resistance: 18
\item
  Degraded Functionality: 15
\item
  Essential Utility: 15
\end{itemize}

Total: 100

\subsection{3. Principle Definitions and Level
Criteria}\label{3-principle-definitions-and-level-criteria}

\subsubsection{3.1 Reversibility (Weight:
16)}\label{31-reversibility-weight-16}

Can destructive actions be undone, and are recovery pathways
non-punitive?

\begin{itemize}
\tightlist
\item
  \texttt{0}: No undo, permanent deletion, no recovery mechanism
\item
  \texttt{1}: Basic undo for UI actions, but no data recovery
\item
  \texttt{2}: Soft deletion + recovery window (7+ days)
\item
  \texttt{3}: State history, export-before-delete, tested recovery flows
\item
  \texttt{4}: Independently verified + user-initiated full restore from
  prior state snapshots
\end{itemize}

\subsubsection{3.2 Exposure Minimization (Weight:
18)}\label{32-exposure-minimization-weight-18}

Is data surface area minimized across network, dependencies, telemetry,
and retention behavior?

\begin{itemize}
\tightlist
\item
  \texttt{0}: Background telemetry, third-party tracking, remote
  dependencies in critical path
\item
  \texttt{1}: Analytics opt-out exists but defaults-on
\item
  \texttt{2}: No analytics; minimal network calls; documented egress
\item
  \texttt{3}: Verified egress allowlist and reproducible network audit
\item
  \texttt{4}: Independent network review with zero unexpected data
  egress
\end{itemize}

\subsubsection{3.3 Local Authority (Weight:
18)}\label{33-local-authority-weight-18}

Can users access and operate core workflows without remote
authentication, permission, or service continuity?

\begin{itemize}
\tightlist
\item
  \texttt{0}: Core functionality requires internet/cloud gatekeeping
\item
  \texttt{1}: Offline mode exists but essential workflows significantly
  degrade
\item
  \texttt{2}: Core workflows operate offline; sync remains optional
\item
  \texttt{3}: User controls export/state portability; offline flows are
  tested and documented
\item
  \texttt{4}: Independently verified local-first architecture with no
  essential cloud lock-in
\end{itemize}

\subsubsection{3.4 Coercion Resistance (Weight:
18)}\label{34-coercion-resistance-weight-18}

Can the system bound disclosure and preserve safety under adversarial
device inspection and pressure?

\begin{itemize}
\tightlist
\item
  \texttt{0}: No panic/discreet controls; obvious labeling and
  persistent revealing notifications
\item
  \texttt{1}: Basic panic/discreet control exists but is weakly tested
\item
  \texttt{2}: Low-visibility mode and discreet notifications with
  internal tests
\item
  \texttt{3}: Reproducible adversarial scenario tests (forced audit /
  seizure simulations)
\item
  \texttt{4}: Independent adversarial review with documented threat
  boundary validation
\end{itemize}

\subsubsection{3.5 Degraded Functionality (Weight:
15)}\label{35-degraded-functionality-weight-15}

Does the system remain usable under resource constraints (network
instability, low battery, constrained compute, cognitive overload)?

\begin{itemize}
\tightlist
\item
  \texttt{0}: Requires stable connectivity/resources for essential use
\item
  \texttt{1}: Graceful errors only; no practical degraded-mode
  continuity
\item
  \texttt{2}: Core flows operate in offline/low-resource conditions
\item
  \texttt{3}: Tested under constrained conditions (for example low
  battery and weak network)
\item
  \texttt{4}: Independent stress-test verification of degraded-mode
  reliability
\end{itemize}

\subsubsection{3.6 Essential Utility (Weight:
15)}\label{36-essential-utility-weight-15}

Are critical-path features available without paywall, institutional
lock, or discretionary gatekeeping?

\begin{itemize}
\tightlist
\item
  \texttt{0}: Paywall or lockout on essential workflows
\item
  \texttt{1}: Freemium model with essential-path restrictions
\item
  \texttt{2}: Core survival/critical features are unrestricted
\item
  \texttt{3}: Verified absence of paywall/lockout on documented
  essential paths
\item
  \texttt{4}: Independent audit confirms no essential-path gatekeeping
\end{itemize}

\subsection{4. Weighted Composite
Calculation}\label{4-weighted-composite-calculation}

For each principle:

\texttt{principle\_points\ =\ (level\ /\ 4)\ *\ weight}

Composite score:

\texttt{PLS\ =\ sum(principle\_points)}

\subsection{5. Hard Fail Guards}\label{5-hard-fail-guards}

Any of the following sets disposition to \texttt{Fail} regardless of
weighted score:

\begin{enumerate}
\def\labelenumi{\arabic{enumi}.}
\tightlist
\item
  Stage 3 gate failure
  (\texttt{WEAK\_VERIFICATION\_COUNT\ \textgreater{}\ 0})
\item
  Evidence of master decrypt/backdoor capability
\item
  Essential workflow paywall or lockout in free/critical path
\item
  Missing threat-boundary disclosure for coercion contexts
\end{enumerate}

\subsection{6. Disposition Bands}\label{6-disposition-bands}

\begin{itemize}
\tightlist
\item
  \texttt{85–100}: Strong legitimacy (operationally reliable)
\item
  \texttt{70–84}: Conditional legitimacy (targeted remediation required)
\item
  \texttt{50–69}: Weak legitimacy (substantial gaps)
\item
  \texttt{\textless{}50}: Non-legitimate under Protective standard
\end{itemize}

\subsection{7. Worked Examples}\label{7-worked-examples}

\subsubsection{7.1 Example: Conventional Cloud Note-Taking
App}\label{71-example-conventional-cloud-note-taking-app}

{\def\LTcaptype{none} % do not increment counter
\begin{longtable}[]{@{}llll@{}}
\toprule\noalign{}
Principle & Level & Rationale & Points \\
\midrule\noalign{}
\endhead
\bottomrule\noalign{}
\endlastfoot
Reversibility & 1 & Basic undo, but deleted notes unrecoverable after
retention window & 4.00 \\
Exposure Minimization & 0 & Analytics telemetry, third-party tracking,
cloud sync dependency & 0.00 \\
Local Authority & 0 & Requires login/cloud for essential state
continuity & 0.00 \\
Coercion Resistance & 0 & No discreet mode, obvious labeling, persistent
notification surface & 0.00 \\
Degraded Functionality & 1 & Network-loss errors without real degraded
continuity & 3.75 \\
Essential Utility & 2 & Free tier exists but essential operations are
constrained & 7.50 \\
\end{longtable}
}

\textbf{Total PLS:} \texttt{15.25\ /\ 100}\\
\textbf{Disposition:} \texttt{Non-legitimate}

\textbf{Hard Fail Guards:}

\begin{itemize}
\tightlist
\item
  Potential authority/paywall constraints on essential path
\item
  No coercion boundary disclosure
\end{itemize}

\begin{center}\rule{0.5\linewidth}{0.5pt}\end{center}

\subsubsection{7.2 Example: PainTracker Reference Mapping
(Illustrative)}\label{72-example-paintracker-reference-mapping-illustrative}

{\def\LTcaptype{none} % do not increment counter
\begin{longtable}[]{@{}llll@{}}
\toprule\noalign{}
Principle & Level & Rationale & Points \\
\midrule\noalign{}
\endhead
\bottomrule\noalign{}
\endlastfoot
Reversibility & 3 & Recovery window and reversibility controls
documented and testable & 12.00 \\
Exposure Minimization & 4 & No analytics, bounded egress assumptions,
strong minimization posture & 18.00 \\
Local Authority & 4 & Offline-capable essential flows and local-control
emphasis & 18.00 \\
Coercion Resistance & 3 & Threat boundaries and coercion scenarios
documented with tests & 13.50 \\
Degraded Functionality & 3 & Essential paths tested under constrained
operation assumptions & 11.25 \\
Essential Utility & 4 & Essential path not paywalled and utility-first
orientation & 15.00 \\
\end{longtable}
}

\textbf{Total PLS:} \texttt{87.75\ /\ 100}\\
\textbf{Disposition:} \texttt{Strong\ legitimacy}

\textbf{Hard Fail Guards:}

\begin{itemize}
\tightlist
\item
  None triggered in this illustrative profile
\end{itemize}

\subsection{8. Anti-Gaming Safeguards}\label{8-anti-gaming-safeguards}

The rubric is designed to resist Goodhart's Law (optimizing score
without protective outcomes).

\begin{enumerate}
\def\labelenumi{\arabic{enumi}.}
\tightlist
\item
  \textbf{No security theater}

  \begin{itemize}
  \tightlist
  \item
    Claims require reproducible evidence, not marketing language.
  \end{itemize}
\item
  \textbf{No hollow reversibility}

  \begin{itemize}
  \tightlist
  \item
    UI-only undo without state/data recovery does not score above
    \texttt{1}.
  \end{itemize}
\item
  \textbf{No fake offline mode}

  \begin{itemize}
  \tightlist
  \item
    Offline claims must preserve essential workflows under real network
    isolation.
  \end{itemize}
\item
  \textbf{No cosmetic coercion controls}

  \begin{itemize}
  \tightlist
  \item
    Discreet/panic features must be validated under adversarial scenario
    tests.
  \end{itemize}
\item
  \textbf{No self-attested Level 4}

  \begin{itemize}
  \tightlist
  \item
    Level \texttt{4} requires independent review evidence.
  \end{itemize}
\end{enumerate}

\subsection{9. Re-Scoring Triggers}\label{9-re-scoring-triggers}

Systems should be re-scored when:

\begin{enumerate}
\def\labelenumi{\arabic{enumi}.}
\tightlist
\item
  Major architectural changes occur (for example sync model or
  dependency shifts)
\item
  Security incidents occur
\item
  Critical-path feature set changes
\item
  Annual audit cycle occurs (minimum every 12 months)
\item
  New community threat evidence is validated
\end{enumerate}

Re-scoring is protective maintenance, not punitive administration.

\subsection{10. Evidence Requirements}\label{10-evidence-requirements}

Minimum evidence set:

\begin{itemize}
\tightlist
\item
  Stage 1/2/3 CI outputs
\item
  MUST-justification ledger rows and implementation statuses
\item
  Threat scenario artifacts (coercion, offline, egress, key-path)
\item
  Verification logs with explicit pass/fail criteria
\end{itemize}

\subsection{11. Reporting Format}\label{11-reporting-format}

A compliant report should include:

\begin{enumerate}
\def\labelenumi{\arabic{enumi}.}
\tightlist
\item
  Commit/version reviewed
\item
  Principle levels (\texttt{0–4}) and rationale
\item
  Weighted total
\item
  Hard fail guard checks
\item
  Required remediation and re-test conditions
\end{enumerate}

\subsection{12. CI/CD Integration Guidance
(Optional)}\label{12-cicd-integration-guidance-optional}

PLS can be partially automated through staged testing pipelines. Stage
labels below map to common test layers and can be adapted to local CI
naming:

\begin{itemize}
\tightlist
\item
  \textbf{Stage 1 (Unit/Component):} reversibility and recovery
  mechanism tests (for example soft delete, undo, export)
\item
  \textbf{Stage 2 (Integration/System):} offline behavior and egress
  conformance checks (for example network isolation, allowlist
  validation)
\item
  \textbf{Stage 3 (Adversarial/E2E):} coercion and degraded-condition
  scenario checks (for example panic-mode inspection, low-battery
  workflows)
\end{itemize}

CI may auto-fail builds on hard-fail guard violations (for example
unexpected network egress, new analytics endpoint, or Stage 3 semantic
failure).

\textbf{However:} Human review remains required for contextual judgment,
threat-model alignment, and final disposition scoring.

\subsection{13. Call for Reviewers}\label{13-call-for-reviewers}

This rubric is pending independent review before promotion to normative
status. Feedback is invited on:

\begin{enumerate}
\def\labelenumi{\arabic{enumi}.}
\tightlist
\item
  \textbf{Principle criteria clarity} --- are level definitions
  (\texttt{0–4}) unambiguous and testable?
\item
  \textbf{Worked example accuracy} --- do scoring examples reflect
  realistic system architectures?
\item
  \textbf{Anti-gaming robustness} --- can safeguards resist
  Goodhart\textquotesingle s Law in practice?
\item
  \textbf{Hard fail guard completeness} --- are critical failure modes
  missing?
\item
  \textbf{Weighting justification} --- do principle weights
  (\texttt{16/18/18/18/15/15}) reflect relative vulnerability cost?
\end{enumerate}

Submit feedback via:

\begin{itemize}
\tightlist
\item
  GitHub Issues:
  \url{https://github.com/protective-computing/protective-computing.github.io/issues}
\item
  Community page: \url{https://protective-computing.github.io/}
\end{itemize}

Recommended review window: 2 weeks from release-candidate publication.

\subsection{14. Version History \&
Feedback}\label{14-version-history--feedback}

\begin{itemize}
\tightlist
\item
  \textbf{v1.0 (Draft):} process guidance; non-normative and pending
  independent review cycle.
\end{itemize}

Promotion to citable artifact should require:

\begin{enumerate}
\def\labelenumi{\arabic{enumi}.}
\tightlist
\item
  Explicit principle criteria validated by reviewers
\item
  Worked examples verified for arithmetic and rationale
\item
  Anti-gaming safeguards reviewed independently
\item
  At least one external reviewer feedback cycle completed
\end{enumerate}

\end{document}
